\documentclass[10pt]{article}
\usepackage[utf8]{inputenc}
\usepackage[hidelinks=true]{hyperref}
\usepackage{xcolor}
\usepackage{url}
\usepackage{geometry}
 \geometry{
 a4paper,
 total={170mm,257mm},
 left=20mm,
 top=20mm,
 }
\hypersetup{
    colorlinks,
    linkcolor={red!50!black},
    citecolor={blue!50!black},
    urlcolor={blue!80!black}
}

\begin{document}

\begin{center}
\Large{
COMP6214 Coursework 1 Using D3.js\\
\url{https://comp6214.herokuapp.com}
}
\\

Shakib-Bin Hamid\\ID: 25250094, Email: sh3g12@soton.ac.uk

\end{center}

\section{Data Cleaning}

\begin{description}
\item[Found at ] \url{https://comp6214.herokuapp.com/cleaning}
\item[On Local Machine] Open \texttt{www/data/odi-cw-1.openrefine.tar.gz} in OpenRefine
\item[Error Correction \& Data Modification] On the same webpage.
\item[Used] OpenRefine\cite{openrefineteam} and MS Excel.
\end{description}

\section{Visualisation 1: Parallel Co-ordinates}

\begin{description}
\item[Found at ] \url{https://comp6214.herokuapp.com/parcoords}
\item[On Local Machine] Open \texttt{www/parcoords.html} file in a browser
\item[Description, Interactivity and Intented Audience] On the same webpage.
\item[Inspired By] \cite{kaichang}
\end{description}

\section{Visualisation 2: Treemap \& Heatmap}

\begin{description}
\item[Found at ] \url{https://comp6214.herokuapp.com/treemap}
\item[On Local Machine] Open \texttt{www/treemap.html} file in a browser
\item[Description, Interactivity and Intented Audience] On the same webpage.
\item[Inspired By] \cite{mikebostock}
\end{description}

\section{Steps of Hosting}

\subsection{On Local Computer}
Make sure to have NodeJS installed. Download the repository. In the repository open a terminal and enter \texttt{
npm install}, followed by \texttt{node index.js}. Go to \url{http://localhost:5000} to see the app.

\subsection{On Heroku}
Detailed steps can be found in \cite{heroku}. In short - install Heroku Toolbelt and log in your Heroku account from a terminal. On Heroku add a NodeJS app and add the app's git address as a remote to the code repository. Then push to the Heroku master branch.

\medskip

\bibliographystyle{unsrt}%Used BibTeX style is unsrt
\bibliography{refs}

\end{document}
